\documentclass{article}
\usepackage[utf8]{inputenc}
\usepackage[T1]{fontenc}
\usepackage{listings}
\usepackage{xcolor}
\usepackage{graphicx}

\graphicspath{{./}}

\lstset{basicstyle=\ttfamily,
  showstringspaces=false,
  commentstyle=\color{red},
  keywordstyle=\color{blue}
}

\title{Zaawansowane interfejsy graficzne - Projekt}
\author{Konrad Tupta, Wojciech Lotko, Kacper Piatkowski, Jaroslaw Skwarek}
\date{\today}

\begin{document}
    
\begin{titlepage}
    \centering
    \vspace*{1in}
    \Huge
    \textbf{Zaawansowane interfejsy graficzne}
    
    \vspace{0.5in}
    \LARGE
    Wypożyczalnia samochodów osobowych 
    
    \vfill
    \Large
    \textbf{Spis członków zespołu:}\\
    Konrad Tupta\\
    Wojciech Lotko\\
    Kacper Piątkowski\\
    Jarosław Skwarek\\
\end{titlepage}

\section{Opis projektu}

\subsection{Funkcjonalności i role}
W projekcie występuje podział na role użytkowników w których skład wchodzą:

\begin{itemize}
    \item Klient
    \item Pracownik
    \item Mechanik
    \item Administrator
\end{itemize}

Każdy użytkownik korzystający z aplikacji ma dostęp do funkjonalności tożsamych z daną rolą przypisaną do jego konta:

\begin{enumerate}
    \item Klient:
    \begin{itemize}
        \item Historia wynajętych samochodów
        \item Rezerwacja samochodu
    \end{itemize}
    \item Pracownik:
    \begin{itemize}
        \item Dostęp do listy wszystkich samochodów w systemie
        \item Dodawanie nowych samochodów
        \item Akceptacja lub odrzucenie prośby o wynajęcie samochodu
        \item Przekazanie samochodu do naprawy
    \end{itemize}
    \item Mechanik:
    \begin{itemize}
        \item Dostęp do listy samochodów do naprawy
        \item Dostęp do historii samochodów do naprawy
        \item Oznaczenie samochodu jako naprawionego i dodanie opisu napraw
        \item Oznaczenie samochodu jako 'nienaprawialnego'
    \end{itemize}
    \item Administrator:
    \begin{itemize}
        \item Dostęp do listy kont użytkowników
        \item Tworzenie kont użytkowników
        \item Modyfikowanie istniejących kont użytkowników
        \item Usuwanie kont użytkowników
    \end{itemize}
\end{enumerate}

\subsection{Wymagania}

Do uruchomienia projektu wymagane są:
\begin{itemize}
    \item SDK .NET 6.0+
    \item MySQL Server lub kompatybilna baza danych
\end{itemize}
\vspace{1em}

\subsection{Instalacja}

\begin{enumerate}
    \item Skolonuj repozytorium i przejdź do katalogu projektu:
    \begin{lstlisting}[language=bash] 
cd CarRent 
    \end{lstlisting}
    \item Przywróć pakiety NuGet:
    \begin{lstlisting}[language=bash] 
dotnet restore
    \end{lstlisting}
\end{enumerate}
\vspace{1em}

\subsection{Konfiguracja}
Aby skonfigurować połączenie z bazą danych, należy ustawić zmienną środowisk-ową, która będzie przechowywać napis do połączenia:

\begin{lstlisting}[language=bash, breaklines=true] 
    # Example
    CAR_RENT_DB_CONNECTION=
    "server=x;database=x;user=x;password=x"
\end{lstlisting}
\vspace{1em}

\subsection{Tworzenie bazy danych}

\begin{enumerate}
    \item Zainstaluj narzędzie Entity Framework Core:
    \begin{lstlisting}[language=bash] 
dotnet tool install --global dotnet-ef
    \end{lstlisting}
    \item Utwórz pierwszą migrację:
    \begin{lstlisting}[language=bash] 
dotnet ef migrations add InitialCreate
    \end{lstlisting}
    \item Zaktualizuj bazę danych
    \begin{lstlisting}[language=bash] 
dotnet ef database update
    \end{lstlisting}
\end{enumerate}
\vspace{1em}

\subsection{Uruchomienie aplikacji}
Aby uruchomić aplikację, należy wykonać polecenia:
\begin{lstlisting}[language=bash] 
    dotnet run
\end{lstlisting}

\vspace{1em}

\subsection{Zależności}
Projekt korzysta z następujących pakietów:

\begin{itemize}
    \item \textbf{Microsoft.EntityFrameworkCore} (v8.0.2)
    \item \textbf{Microsoft.EntityFrameworkCore.Design} (v8.0.2)
    \item \textbf{Microsoft.EntityFrameworkCore.SqlServer} (v8.0.2)
    \item \textbf{Microsoft.EntityFrameworkCore.Tools} (v8.0.2)
    \item \textbf{MySql.EntityFrameworkCore} (v8.0.0)
    \item \textbf{Pomelo.EntityFrameworkCore.MySql} (v8.0.0)
\end{itemize}

\section{Jak używać aplikacji}

Funkcjonalności podzielone zostały ze względu na rolę użytkownika. Aby mieć do nich dostęp, należy zalogować się do odpowiadających ról.

\subsection{Administrator}
\vspace{1em}

\textbf{Funkcjonalności:}
\vspace{1em}

\textbf{Dodanie użytkownika:}

\begin{enumerate}
    \item Wypełnić dane \textit{Firstname, Surname, Email, Role}
    \item Nacisnąć przycisk 'Add user'
\end{enumerate}

\textbf{Modyfikacja użytkownika:}

\begin{enumerate}
    \item Otworzyć okno kontekstowe użytkownika za pomocą prawego przycisku myszy
    \item Wybrać opcję 'Edit'
    \item Zmodyfikować informacje użytkownika
    \item Nacisnąć przycisk 'Save'
\end{enumerate}

\textbf{Usunięcie użytkownika:}

\begin{enumerate}
    \item Otworzyć okno kontekstowe użytkownika za pomocą prawego przycisku myszy
    \item Wybrać opcję 'Delete'
    \item Potwierdzić chęć usunięcia użytkownika
\end{enumerate}

\subsection{Pracownik}
\vspace{1em}

\textbf{Funkcjonalności:}
\vspace{1em}

\textbf{Dodanie samochodu:}

\begin{enumerate}
    \item Wypełnić dane \textit{Make, Model, Year, Horse Power, Car State}
    \item Nacisnąć przycisk 'Add Car'
\end{enumerate}

\textbf{Akceptacja rezerwacji samochodu:}

\begin{enumerate}
    \item Pzejść do zakładki 'Requests'
    \item Wybrać wiersz z prośbą o rezerwację / wynajęcie
    \item Nacisnąć przycisk 'Reserve'
\end{enumerate}

\textbf{Wynajęcie samochodu:}

\begin{enumerate}
    \item Pzejść do zakładki 'Requests'
    \item Wybrać wiersz z prośbą o rezerwację / wynajęcie
    \item Nacisnąć przycisk 'Rent'
\end{enumerate}

\textbf{Odrzucenie rezerwacji samochodu:}

\begin{enumerate}
    \item Pzejść do zakładki 'Requests'
    \item Wybrać wiersz z prośbą o rezerwację / wynajęcie
    \item Nacisnąć przycisk 'Reject'
\end{enumerate}

\textbf{Przekazanie samochodu do mechanika:}

\begin{enumerate}
    \item Pzejść do zakładki 'Requests'
    \item Wybrać wiersz z samochodem w dowolnym stanie wynajęcia
    \item Nacisnąć przycisk 'Send to service'
\end{enumerate}

\subsection{Mechanik}

\textbf{Funkcjonalności:}
\vspace{1em}

\textbf{Zmiana stanu samochodu na naprawiony:}

\begin{enumerate}
    \item Wybrać wiersz z samochodem do naprawy
    \item Nacisnąć przycisk 'Set as repaired'
    \item W oknie kontekstowym wypełnić pola \textit{Description, Cost, Quantity}
    \item Potwierdzić przyciskiem 'OK'
\end{enumerate}

\begin{enumerate}
    \item Wybrać wiersz z samochodem do zmiany statusu
    \item Nacisnąć przycisk 'Set as unavailable'
    \item Potwierdzić przyciskiem 'OK'
\end{enumerate}

\subsection{Klient}
\textbf{Funkcjonalności:}
\vspace{1em}

\textbf{Prośba o wynajęcie samochodu:}

\begin{enumerate}
    \item Wybrać wiersz z samochodem do wynajęcia
    \item Wypełnić datę odbioru i zdania samochodu
    \item Nacisnąć przycisk 'Rent'
\end{enumerate}

\subsection{Przykładowy scenariusz}
\vspace{1em}

\begin{enumerate}
    \item Administrator dodaje użytkownika z rolami pracownika i mechanika 
    \item Klient tworzy swoje własne konto za pomocą panelu rejestracyjnego
    \item Pracownik dodaje samochód w aplikacji
    \item Klient wysyła prośbę o zaakceptowanie wynajęcia auta
    \item Pracownik akceptuje wynajęcie, stan auta zostaje zmieniony na 'Reserved'
    \item Gdy klient odbiera klucze do auta w salonie, pracownik zmienia stan auta na 'Rented'
\end{enumerate}

\end{document}